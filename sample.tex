% !Mode:: "TeX:UTF-8"
%# -*- coding:utf-8 -*-

%% 南京大学学位论文的示例文档
%% 作者:njuhan: https://github.com/njuHan
%% 源模版repo: https://github.com/njuHan/njuthesis-nju-thesis-template

\documentclass[winfonts,master,twoside,AutoFakeBold= {2}]{njuthesis}
%% njuthesis 文档类的可选参数有:
%%   winfonts, linuxfonts, macfonts, adobefonts winfonts 选项使得文档使用Windows 系统提供的字体;linuxfonts 选项使得文档使用Linux 系统提供的字体;macfonts 选项使得文档使用Mac 系统提供的字体;adobefonts 选项使得文档使用Adobe提供的OTF中文字体(需自行下载安转)
%%   phd/master/bachelor 选择博士/硕士/学士论文
%%   twoside 或 oneside 指定排版的文档为双面打印或单面打印格式(twoside会使得chapter 章节从奇数页开始,即纸张的正面开始,因此会出现一些空白的页面)
%%   nobackinfo 取消封二页导师签名信息。注意,按照南大的规定,是需要签名页的。
%%   AutoFakeBold 设置CJK字体加粗,参数“2”用于指定加粗程度(空格勿删,删除会引发编译错误)


%%%%%%%%%%%%%%%%%%%%%%%%%%%%%%%%%%%%%%%%%%%%%%%%%%%%%%%%%%%%%%%%%%%%%%%%%%%%%%%
% set up labelformat and labelsep for subfigure 详见: http://www.latexstudio.net/archives/8652.html
\captionsetup[subfigure]{labelformat=simple, labelsep=space}

%%%%%%%%%%%%%%%%%%%%%%%%%%%%%%%%%%%%%%%%%%%%%%%%%%%%%%%%%%%%%%%%%%%%%%%%%%%%%%%
% 设置《国家图书馆封面》的内容,仅博士论文才需要填写

% 设置论文按照《中国图书资料分类法》的分类编号
\classification{0175.2}
% 设置论文按照《国际十进分类法UDC》的分类编号
% 该编号可在下述网址查询:https://udcsummary.info/php/index.php?lang=chi
\udc{004.72}
% 国家图书馆封面上的论文标题第一行,不可换行。此属性可选,默认值为通过\title设置的标题。
\nlctitlea{论文标题第一行}
% 国家图书馆封面上的论文标题第二行,不可换行。此属性可选,默认值为空白。
\nlctitleb{论文标题第二行}
% 国家图书馆封面上的论文标题第三行,不可换行。此属性可选,默认值为空白。
\nlctitlec{}
% 导师的单位名称及地址
\supervisorinfo{南京大学计算机科学与技术系~~南京市汉口路22号~~210093}
% 答辩委员会主席
\chairman{张三丰~~教授}
% 第一位评阅人
\reviewera{阳顶天~~教授}
% 第二位评阅人
\reviewerb{张无忌~~副教授}
% 第三位评阅人
\reviewerc{黄裳~~教授}
% 第四位评阅人
\reviewerd{郭靖~~研究员}


%%%%%%%%%%%%%%%%%%%%%%%%%%%%%%%%%%%%%%%%%%%%%%%%%%%%%%%%%%%%%%%%%%%%%%%%%%%%%%%
% 设置论文的中文封面

% 单行论文标题,不可换行
\title{南京大学毕业论文\LaTeX 模板}

% 如果论文标题过长,可以分两行,第一行用\titlea{}定义,第二行用\titleb{}定义,
% 使用以下3行:
%\title{} %用于覆盖单行标题内容为空
%\titlea{长标题第一行}  %第一行标题写这里
%\titleb{长标题第二行用于长标题换行} %第二行标题写这里
% 注意: \title 不能都注释,它用于控制标题选择双行还是单行。\title{}如果内容为空,则编译\titlea{},titleb{}双行标题,否则编译单行标题


% 论文作者姓名
\author{管俣祺}
% 论文作者联系电话
\telphone{xxxx}
% 论文作者电子邮件地址
\email{sample@smail.nju.edu.cn}
% 论文作者学生证号
\studentnum{MF20330024}
% 论文作者入学年份(年级)
\grade{2012}
% 论文作者毕业年份(届), 出版授权书的学位年度
\graduateyear{20xx}
% 导师姓名职称
\supervisor{申富饶~~教授}
% 导师的联系电话
\supervisortelphone{}
% 论文作者的学科与专业方向
\major{计算机科学与技术}
% 论文作者的研究方向
\researchfield{人工智能}
% 论文作者所在院系的中文名称
\department{计算机科学与技术系}
% 论文作者所在学校或机构的名称。此属性可选,默认值为``南京大学''。
\institute{南京大学}
% 论文的提交日期,需设置年、月、日。
\submitdate{xxxx年 xx 月 xx 日}
% 论文的答辩日期,需设置年、月、日。
\defenddate{xxxx年 xx 月 xx 日}
% 论文的定稿日期,需设置年、月、日。
% 此属性可选,若注释\date{},则默认值为最后一次编译时的日期,精确到日。
% \date{2019年5月20日}

%%%%%%%%%%%%%%%%%%%%%%%%%%%%%%%%%%%%%%%%%%%%%%%%%%%%%%%%%%%%%%%%%%%%%%%%%%%%%%%
% 设置论文的英文封面

% 论文的英文标题,不可换行
\englishtitle{\LaTeX \;  Post-processing Algorithm for Video Object Detection Based on Context Information }
% 论文作者姓名的拼音
\englishauthor{Authoraaaa}
% 导师姓名职称的英文
\englishsupervisor{ Professor}
% 论文作者学科与专业的英文名
\englishmajor{Computer Science and Technology}
% 论文作者所在院系的英文名称
\englishdepartment{Department of Computer Science and Technology}
% 论文作者所在学校或机构的英文名称。此属性可选,默认值为``Nanjing University''。
\englishinstitute{Nanjing University}
% 论文完成日期的英文形式,它将出现在英文封面下方。需设置年、月、日。日期格式使用美国的日期
% 格式,即``Month day, year'',其中``Month''为月份的英文名全称,首字母大写;``day''为
% 该月中日期的阿拉伯数字表示;``year''为年份的四位阿拉伯数字表示。
% 此属性可选,若注释掉\englishdate{},则默认值为最后一次编译时的日期。
% \englishdate{May 20, 2019}

%%%%%%%%%%%%%%%%%%%%%%%%%%%%%%%%%%%%%%%%%%%%%%%%%%%%%%%%%%%%%%%%%%%%%%%%%%%%%%%
% 设置论文的中文摘要

% 设置中文摘要页面的论文标题及副标题的第一行。
% 此属性可选,其默认值为使用|\title|命令所设置的论文标题
\abstracttitlea{标题第一行}
% 设置中文摘要页面的论文标题及副标题的第二行。
% 此属性可选,其默认值为空白
\abstracttitleb{标题第二行用于长标题换行}

%%%%%%%%%%%%%%%%%%%%%%%%%%%%%%%%%%%%%%%%%%%%%%%%%%%%%%%%%%%%%%%%%%%%%%%%%%%%%%%
% 设置论文的英文摘要

% 设置英文摘要页面的论文标题及副标题的第一行。
% 此属性可选,其默认值为使用|\englishtitle|命令所设置的论文标题
\englishabstracttitlea{englishabstracttitlea}
% 设置英文摘要页面的论文标题及副标题的第二行。
% 此属性可选,其默认值为空白
\englishabstracttitleb{nglishabstracttitleb}

%%%%%%%%%%%%%%%%%%%%%%%%%%%%%%%%%%%%%%%%%%%%%%%%%%%%%%%%%%%%%%%%%%%%%%%%%%%%%%
%% 盲审命令,空白字段设置请看 .cls文件 \newcommand*{\blind}
%% 此外,请按照盲审要求自行去掉个人简历、致谢等页面中的个人信息
%\blind

%%%%%%%%%%%%%%%%%%%%%%%%%%%%%%%%%%%%%%%%%%%%%%%%%%%%%%%%%%%%%%%%%%%%%%%%%%%%%%%
\begin{document}

%%%%%%%%%%%%%%%%%%%%%%%%%%%%%%%%%%%%%%%%%%%%%%%%%%%%%%%%%%%%%%%%%%%%%%%%%%%%%%%

% 制作国家图书馆封面(博士学位论文才需要)
%\makenlctitle
% 制作中文封面
\maketitle
% 制作英文封面
\makeenglishtitle


%%%%%%%%%%%%%%%%%%%%%%%%%%%%%%%%%%%%%%%%%%%%%%%%%%%%%%%%%%%%%%%%%%%%%%%%%%%%%%%
% 开始前言部分
\frontmatter

%%%%%%%%%%%%%%%%%%%%%%%%%%%%%%%%%%%%%%%%%%%%%%%%%%%%%%%%%%%%%%%%%%%%%%%%%%%%%%%
% 论文的中文摘要
\begin{abstract}
\lipsum[1-2]

%通过改变链路中子流的个数,分配不同的数据流量给不同的链路。

% 中文关键词。关键词之间用中文全角分号隔开,末尾无标点符号。
\keywords{关键词1 \quad 关键词2 }
\end{abstract}

%%%%%%%%%%%%%%%%%%%%%%%%%%%%%%%%%%%%%%%%%%%%%%%%%%%%%%%%%%%%%%%%%%%%%%%%%%%%%%%
% 论文的英文摘要
\begin{englishabstract}
%\lipsum[2]
wwwwwwwwwwwwww
%Rate adaptation can be implemented by adjusting the number of subflows on each path.

% 英文关键词。关键词之间用英文半角逗号隔开,末尾无符号。
\englishkeywords{keyword1\quad keyword2}
\end{englishabstract}

%%%%%%%%%%%%%%%%%%%%%%%%%%%%%%%%%%%%%%%%%%%%%%%%%%%%%%%%%%%%%%%%%%%%%%%%%%%%%%%
% 论文的前言,应放在目录之前,中英文摘要之后
%
\begin{preface}
\lipsum[1]
\vspace{1cm}
\begin{flushright}
作者\\
20xx年夏于南京大学
\end{flushright}

\end{preface}

%%%%%%%%%%%%%%%%%%%%%%%%%%%%%%%%%%%%%%%%%%%%%%%%%%%%%%%%%%%%%%%%%%%%%%%%%%%%%%%
% 生成论文目录
\tableofcontents

%%%%%%%%%%%%%%%%%%%%%%%%%%%%%%%%%%%%%%%%%%%%%%%%%%%%%%%%%%%%%%%%%%%%%%%%%%%%%%%
% 生成插图清单。如无需插图清单则可注释掉下述语句。
\listoffigures

%%%%%%%%%%%%%%%%%%%%%%%%%%%%%%%%%%%%%%%%%%%%%%%%%%%%%%%%%%%%%%%%%%%%%%%%%%%%%%%
% 生成附表清单。如无需附表清单则可注释掉下述语句。
\listoftables

%%%%%%%%%%%%%%%%%%%%%%%%%%%%%%%%%%%%%%%%%%%%%%%%%%%%%%%%%%%%%%%%%%%%%%%%%%%%%%%
% 开始正文部分
\mainmatter

%%%%%%%%%%%%%%%%%%%%%%%%%%%%%%%%%%%%%%%%%%%%%%%%%%%%%%%%%%%%%%%%%%%%%%%%%%%%%%%
% 学位论文的正文应以《绪论》作为第一章
\chapter{绪论}\label{chapter_introduction}
\section{研究背景}
近年来,基于信息时代的海量数据以及日益提高的数据处理能力和计算能力,机器学习领域的发展尤为迅猛,特别是深度学习方面,越来越多的基于深度学习的应用出现在我们的日常生活中,降低了很多场景中的人工成本,比较典型的一些应用是计算机视觉、自然语言处理、语音信号处理、推荐系统、时间序列分析、对抗攻击安全研究等。其中,计算机视觉是较早被应用于我们日常生活的一种技术,具体来说,其任务是训练计算机和系统,使之能够从视觉输入信息中提取出用户所需要的有意义的目标信息,并在此基础上进行后续的分析并反馈给用户。计算机视觉基于大量的图像和视频等数据基础,结合以卷积神经网络 (ConvolutionalNeuralNetworks,CNN)为例的网络结构,实现像素级的内容处理,最终实现让机器和系统模拟人类视觉系统工作机制来处理图像画面。图像数据可以采用多种形式,比如静态图像数据、动态视频序列、多个视角的摄像机图像数据以及一些通过扫描得到的多维数据,而计算机视觉技术关注的正是基于图像数据,将学科理论和模型应用于计算机视觉系统的构建。

上述内容简要介绍了计算机视觉任务目标及基础原理,而图像内容检测、目标检测、图像分类、内容跟踪、3D场景重建等均属于比较常见的计算机视觉任务。其中,目标检测是计算机视觉领域最重要也最具有挑战性的分支之一,主要任务是检测数字图像和视频中某一类或者多类语义对象(例如汽车、动物、行人等),因此目标检测在我们日常生活中的应用也非常普遍,如自动驾驶、视频监控、场景分析、机器人视觉等。近两年也因为目标检测日益广泛的应用和快速的技术突破而受到学术界和商业界越来越多的关注。目标检测的实现方式通常分为基于神经网络的方法及非神经网络方法。对于非神经网络方法,通常需要预先定义特征然后再利用以支持向量机(SVM)为例的技术进行分类。而另一种基于神经网络的方法,则能够做到在不具体定义特征的情况下进行端到端的目标对象检测,这一类方法通常基于卷积神经网络(CNN)来实现,RCNN是第一个成功将深度学习应用于目标检测上的算法,在此基础上也不断有新的目标检测模型被提出并应用到这一任务领域内,本文将要描述的后处理方法,也正是基于这类基于神经网络的目标检测方法所提出的优化机制。

目标检测的技实现中,后处理是十分重要的一环,在前文提到的基于神经网络实现目标检测的方法中,以RCNN为例的方法主要利用选择性搜索来生成候选区域,再进行特征提取和分类,最后利用NMS这类方法得到修正之后的检测结果。目标检测后处理方案主要是对修正步骤的优化,前序模型检测结果包含若干个目标候选框,后处理的任务是对这些目标候选框进行筛选和过滤,得到最终的目标检测结果框。另外,目标检测领域进一步细分,可以分为基于静态图片的目标检测和基于视频流的目标检测,其中基于视频流的目标检测, 由于涉及到视频时序信息及上下文信息,在后处理阶段可以设计的内容就更多,相应的后处理部分也就能提供更好的筛选过滤效果,从整体上达到提高目标检测精度的效果。

综上所述,视频目标检测后处理作为视频目标检测的技术补充,可以在计算代价低的前提下实现对视频目标检测的精度提升,从而更好的将静态图像目标检测模型用于视频目标检测的场景下,是一种高效的解决方案,具有重要的研究意义。


\section{研究现状}\label{subsec:mptcp_conges}
本文主要工作是围绕视频目标检测的后处理方案展开,目标检测器的输入数据一般为静态图像或者视频流,可以粗略分为静态图像检测器和视频检测器两种。而视频目标检测有两种主流的解决方案, 一种是将应用于静态图像的目标检测模型逐帧的应用于视频流,再结合后处理机制对视频数据的上下文信息加以利用,另一种则是基于视频级的数据和特征工作训练专用于视频的目标检测模型。本节内容主要介绍目标检测及视频目标检测后处理的研究现状。
\subsection{目标检测}
目标检测是一项重要的计算机视觉任务,主要目的是检测数字图像中特定类别的视觉对象,这一技术的应用十分广泛,包括辅助驾驶、智能避障、视频监控、图像检索等场景。在过去的几年里,深度学习技术的快速发展大大加速了目标检测的发展进程,借助深度学习网络和GPU的计算能力,目标检测的性能得到了很大的提升,也奠定了许多下游计算机视觉任务的基础,例如图像字幕、对象跟踪等。在目标检测领域的研究初期,一般研究的是针对于图像数据的目标检测,后期有了针对于视频数据的目标检测,因此又把针对于图像数据的目标检测器称为静态图像目标检测器或通用检测器,把针对于视频数据的检测器称为视频目标检测器。

事实上,目标检测,也就是静态图像目标检测这一任务的提出和发展并不只是近几年的研究内容,其发展进程通常分为两个时期,以引入深度学习作为分割,分为之前和之后两个阶段。2014年以前主要是利用传统技术手段进行目标检测,2001年提出的Viola-Jones检测器开创了目标检测发展的先河,2006年的HOG检测器在计算机视觉和图像处理任务中使用了特征描述符,而2008年的DPM首次引入了边界框回归。从2014年开始,以RCNN为首的检测器将目标检测带入了深度学习检测的时代,后续的检测器被分为two-stage和one-stage两个类型。通常来说,基于深度学习的对象检测器从输入图像或者视频帧中提取特征,再解决后续两个任务:查找存在于画面中的任意数量的目标;对每个目标进行分类并使用边界框预测每个目标的大小和位置信息。two-stage类型的检测器分阶段处理这两个任务,相应的,one-stage检测器选择将这些任务合并为一个步骤来执行,以牺牲一定的准确性为代价来获得更高的性能。如前文所述,卷积神经网络CNN的引入极大的促进了目标检测领域的发展,许多基于静态图像的检测模型,如R-CNN及其变体、YOLO及其变体,都为静态图像目标检测建立了良好且有效的检测框架。目前较为主流的静态图像目标检测模型,其处理机制可以概括为三个阶段:建议区域生成阶段;目标分类阶段;后处理阶段。首先,在建议区域生成阶段,根据图像区域中包含目标的可能性大小,来生成一组候选区域,更早之前的建议区域生成方法是基于更低层次的图像特征,但目前的以Faster R-CNN为例的静态图像目标检测模型则是使用神经网络来学习并生成建议区域。随后在第二个目标分类阶段,则是给候选区域分配一个等级分数。最后在后处理阶段对多个目标检测框进行冗余过滤,得到最终的反馈结果。

静态图像目标检测在近几年的发展中虽然得到了较大性能上的提升,但将静态图像目标检测模型直接用于视频目标检测则效果欠佳,这是因为静态图像目标检测模型忽略了视频流中时间维度的信息。在视频数据中,前一帧中很容易被检测出且置信度高的目标,在下一帧中可能很难被检测到或者置信度很低,视频画面存在遮挡、剧烈运动、模糊都有可能导致这一情况,这也就使得直接将静态图像目标检测模型应用于视频流变得更为困难。为了解决这一问题,研究者们主要探索出了两种策略。具有高计算成本的针对于视频数据的视频检测器,以及静态图像目标检测器结合快速后处理算法,这两种策略共同构成了目前视频目标检测的主流解决方案。一方面是基于视频数据的特点专门为视频目标检测设计检测模型,这一方案需要在视频数据的基础上,根据当前处理的帧及附近的帧实现特征聚合,并实现与静态目标检测相同或者更高的检测精度,而这需要大量的计算,就导致其检测速度缓慢,则无法很好的适用于低性能设备或者需要实时计算的使用场景。另一方面就是结合后处理方案,后处理方案的输入一般为目标检测模型中已检测出但未被过滤的检测结果,通过帧间关联检测对象,并使用得到的关联信息来优化检测结果,这一类方案由于计算代价低且设计逻辑更为清晰,更适用于工业生产及日常检测的场景。

\subsection{视频目标检测后处理}
视频目标检测与静态图像检测很大的一个不同点是,视频的关键因素是时间信息和上下文信息,因为视频中物体的位置和外观在时间上应该是一致的,换言之,在同一个视频片段内的检测结果在检测框位置和检测框置信度方面不应该发生剧烈变化。这也就使得性能良好的静态图像检测器直接运用于视频上,会因为无法利用这些时间信息和上下文信息而出现性能下滑的情况。另一方面,视频数据中存在物体遮挡、剧烈运动、模糊、罕见角度或姿势,这也加大了视频目标检测的难度,此时视频的上下文信息和时间信息就显得尤为重要。基于计算性能和研发成本的考虑,通过后处理机制弥补静态图像检测模型无法利用时间信息和上下文信息的问题,是一种在成本和收益方面都比较理想的解决方案。

后处理方案的提出是为了将静态目标检测器更好的运用于视频数据但又不在网络模型方面增加过多的负担,这类后处理方法应用于图像检测器每一帧的检测结果,结合时间信息和上下文信息,给出最终的预测输出。后处理方案不仅提高了静态检测器直接运用于视频数据的准确性,同时在计算代价和开发成本上也小于基于视频数据设计视频目标检测器所需要的代价和成本。

常见的后处理方法可以被分成两类,一类是建立帧之间的关联,结合时间信息来进行后处理;另一类则是从特征工程出发,通过视频数据和特征工作来优化检测结果。一般来说,基于特征工作的方法需要的人工设计部分比较少,也可能在准确率方面取得更好的结果,但不得不考虑的问题是,基于特征工作的后处理方法需要视频数据集进行训练,与传统的图像数据集相比,视频数据集收集和标注所需的工作量更大,因此可用的公开数据集也少之又少。因此,基于帧间关联的后处理方法仍然十分重要,使用常见的图像数据集就可以完成训练,因而在实际工程项目中使用的也更多,是一种成效理想且实现代价较低的后处理方案。现有的基于帧间关联的方法大多依赖光流、物体跟踪或跨时间排序,但显而易见的是,基于计算复杂性的考虑,光流的计算成本更高,而且光流分析本身与目标检测这一主线任务没有太直接的关系,仅仅只是作为后处理的一种辅助手段,因此付出太过高昂的计算代价性价比很低。同理,使用目标跟踪手段也会使得计算开销变大,在获得准确率提升的同时却降低了检测效率。与上述问题相似的是,其他很多基于帧间关联的后处理方案都存在计算代价高,不能应用于在线视频检测的问题,但回溯到目标检测的常见应用场景,自动驾驶、视频监控、机器人视觉等诸多任务都是需要实时视频流检测及分析的,少有利用本地视频文件做分析的应用场景,因此基于帧间关联实现后处理的方案也仍有需要优化的地方。
\section{本文研究内容}
本文主要研究应用于视频目标检测的后处理方案,其目的是在静态目标检测模型的基础上,通过后处理方案的设计和实现,将静态目标检测模型过渡并应用于视频目标检测。此类方案主要利用视频检测场景中大量的上下文信息和时间信息,弥补了静态检测模型的不足,并且对于不同的静态目标检测模型具有普适性。除此之外,本文所研究的后处理方案所需要的计算成本较低,是一种高性价比的目标检测优化策略。更进一步的,本文对检测结果的优化分为非实时检测和实时检测两种,既满足了以高精度为目标的非实时检测使用场景,也满足了强调实时性的在线场景。最后,本文所设计的策略应用于实际系统工程中,通过实际效果检验其有效性和实用性。文本的主要研究内容总结如下:
\begin{itemize}
\item 本文利用视频流中的时空信息和上下文信息,设计了一种能应用于静态目标检测模型或视频目标检测模型的后处理策略(Context Information Based Post-Processing,CIBPP)。CIBPP通过位置信息、语义信息、外观信息来描述一个目标检测框,并通过距离函数计算相邻视频帧之间任意两个检测框之间的距离,由此建立距离矩阵,实现跨帧的检测框连接。在建立检测框连接的同时,增加长时反馈机制,将长时间稳定出现的检测框认为是正确的目标检测结果。建立起跨帧的检测框连接之后,每一段连接内通过类分数重置及补偿来优化该视频片段内的误检情况,与此同时在不同的两段连接之间,通过双向扩散和检测框匹配来优化漏检情况。该策略在ImageNet VID数据集上的实验结果证明了其有效性,并且该方法具有普适性,并不限制具体的目标检测模型选择,因此是一种成本低、效果好的目标检测优化策略。
\item 基于上述非实时性视频目标检测后处理策略的设计和实现,更进一步的,本文还提出了一种适用于在线实时检测场景的后处理策略,并使用卡尔曼滤波对检测结果进行平滑(Online Post-processing and Smooth,OPPSmooth)。此类方法在处理每一帧时,仅利用当前帧及前序帧的信息,建立信息队列,在队列内对前向若干帧的检测结果进行连接,并实时重置检测框的位置、类分数等信息。并在一定程度上通过卡尔曼滤波来平衡由于像素信息引起的相邻帧之间检测结果的抖动,以达到稳定检测框的效果。和其他的在线视频目标检测后处理方案相比,本文所提方法取得了更好的效果,并且能在实时检测场景下得到更稳定的检测框输出。
\item 本文将所提方法融合在了实际的系统工程中,该系统场景为井下煤矿作业场景,通过实时目标检测识别出井下的工作人员,同时标识出电缆槽的位置作为安全工作区域和危险工作区域的分界,判断当前是否有工作人员处于危险位置,若有则及时给出预警。该系统可以上传视频,进行非实时本地视频检测,也可以进行在线实时处理,即模拟井下摄像头实时输入的效果,整个系统很好的体现了本文所提的后处理方法在实际应用中的价值。
\end{itemize}

\section{本文结构安排}
本文主要研究的是利用视频上下文信息设计实现后处理策略,使得视频目标检测的准确度得到提升,除了聚焦于准确度的提升,本文同样关注在线检测的实时性要求,又提出了满足实时检测要求的另一种后处理方案,并最终将所提策略应用于实际的工业系统应用中。全文共有六个章节,第一章绪论部分,主要阐述本文的研究背景,从机器学习到目标检测,再到视频目标检测及后处理,一步步细化。同时还介绍了目标检测及后处理这两个领域的研究现状。最后点明本文研究内容;第二章为相关工作部分,分图像目标检测及视频目标检测两个方向介绍相关工作,对一些具有代表性的研究工作进行分析;第三章介绍了本文提出的基于上下文信息的视频目标检测后处理策略,包括问题分析、详细设计、结果分析等内容;第四章主要介绍在线视频目标检测后处理方案,分析了实时检测这一条件的限制下,如何通过后处理实现目标检测结果优化,最后通过测试验证其有效性;第五章介绍了一个融合了前文所述的后处理策略的实际系统工程,并对整个系统的使用方式和功能进行分析,充分说了文本所提方法的实际应用价值;第六章为本文的总结部分,也包含对后续工作及研究方向的展望。

\chapter{相关工作}
本章主要介绍本文所涉及到的相关背景知识,以及国内外一些具有代表性的工作内容。首先介绍图像目标检测的相关内容,包括任务描述、发展历程、代表工作以及最新的处理模式。然后是由图像目标检测进一步引申的视频目标检测这一领域,说明视频目标检测任务的提出,以及主流的解决方案。视频目标检测的解决方案其中之一就是静态目标检测器结合后处理机制,因此对目前常见的具有代表性的一些后处理机制也进行了介绍,为后续章节中提出的后处理方案提供一定的理论基础和背景铺垫。 
\section{图像目标检测}
\subsection{任务描述及性能度量}
早期的目标检测代指的即为图像目标检测,目标检测是计算机视觉的一个重要分支。图像目标检测的任务是对图像中的物体进行识别和定位,通常包括以下步骤:首先对图像或视频进行预处理,如去噪、归一化等;接着通过特征提取算法,提取图像中的关键特征,如 SIFT、HOG 等;然后通过目标检测模型,对图像中的物体进行识别和定位,比如使用 YOLO、Faster R-CNN 等模型;最后对结果进行后处理,比如移除重叠的框,保留最可能的框。目标检测在许多领域都有广泛应用,如自动驾驶、安防监控、人脸识别等。它涉及到计算机视觉、机器学习和图像处理等多领域,是一个具有挑战性的任务。


\section{视频目标检测}
\section{后处理策略}

\chapter{基于视频上下文信息的后处理框架}
\section{问题分析}
\section{基于视频上下文信息的后处理策略}
\section{实验与分析}
\section{本章小结}

\chapter{实时视频目标检测结果平滑及后处理框架}
\section{问题分析}
\section{实时视频目标检测后处理策略}
\section{实时视频目标检测结果的平滑处理}
\section{实验与分析}
\section{本章小结}


\chapter{基于上下文信息的后处理策略及实时后处理策略在系统中的应用}
\section{相关背景}
\section{系统需求}
\section{系统架构}
\section{系统实现}
\section{效果展示}
\section{本章小结}

\chapter{总结与展望}

\chapter{使用示例}
\section{示例如下}
使用.bib文件管理参考文献引用,引用示例:\cite{BHR12}.\par
\begin{itemize}
\item 一级item
 \begin{itemize}
 \item 二级item
	\begin{itemize}
	\item 三级item

	\end{itemize}

 \end{itemize}
\item 一级item

\end{itemize}

\begin{algorithm}[htbp]
  \caption{算法名字}
  \label{alg:alg1}
  \begin{algorithmic}[1]
        \REQUIRE 这是输入
        \ENSURE 这是输出
        \WHILE {flag}
		      \STATE 这是语句
        \ENDWHILE
  \end{algorithmic}
\end{algorithm}

\begin{figure}[htbp]
  \centering
  \includegraphics[width=0.6\linewidth]{./figure/github.jpg}
  \caption{单图示例}
  \label{fig:system}
\end{figure}

实验硬件设备如图\ref{img:1}所示。
\begin{figure}[htbp]
\begin{minipage}[t]{0.5\textwidth}
\centering
\includegraphics[width=0.8\textwidth]{./figure/github.jpg}
\caption{实验硬件设备总览}
\label{img:1}
\end{minipage}
\begin{minipage}[t]{0.5\textwidth}
\centering
\includegraphics[width=0.8\textwidth]{./figure/github.jpg}
\caption{实验测量示意图}
\label{img:2}
\end{minipage}
\end{figure}

图\ref{fig:sub}所示子图\ref{subfig:a}和子图\ref{subfig:b}。
\begin{figure}[H]
	\begin{subfigure}{.5\textwidth}
		\centering
		\includegraphics[width=0.8\textwidth]{./figure/github.jpg}
		\caption{子图}
		\label{subfig:a}
	\end{subfigure}
	\begin{subfigure}{.5\textwidth}
		\centering
		\includegraphics[width=0.8\textwidth]{./figure/github.jpg}
		\caption{子图}
		\label{subfig:b}
	\end{subfigure}
\caption{子图样例}
\label{fig:sub}
\end{figure}




%%%%%%%%%%%%%%%%%%%%%%%%%%%%%%%%%%%%%%%%%%%%%%%%%%%%%%%%%%%%%%%%%%%%%%%%%%%%%%%
% 参考文献。应放在\backmatter之前。
% 推荐使用BibTeX,若不使用BibTeX时注释掉下面一句。
%\nocite{*}
\bibliography{sample}


% 附录,必须放在参考文献后,backmatter前
\appendix
\chapter{附录代码}\label{app:1}
\section{main函数}
\begin{lstlisting}[language=C]
int main()
{
	return 0;
}
\end{lstlisting}


%%%%%%%%%%%%%%%%%%%%%%%%%%%%%%%%%%%%%%%%%%%%%%%%%%%%%%%%%%%%%%%%%%%%%%%%%%%%%%%
% 致谢
\begin{acknowledgement}
%thanks
    \lipsum[1]

\end{acknowledgement}


%%%%%%%%%%%%%%%%%%%%%%%%%%%%%%%%%%%%%%%%%%%%%%%%%%%%%%%%%%%%%%%%%%%%%%%%%%%%%%%
% 书籍附件
\backmatter
%%%%%%%%%%%%%%%%%%%%%%%%%%%%%%%%%%%%%%%%%%%%%%%%%%%%%%%%%%%%%%%%%%%%%%%%%%%%%%%
% 作者简历与科研成果页,应放在backmatter之后
\begin{resume}
% 论文作者身份简介,一句话即可。
\begin{authorinfo}
\noindent 韦小宝,男,汉族,1985年11月出生,江苏省扬州人。
\end{authorinfo}
% 论文作者教育经历列表,按日期从近到远排列,不包括将要申请的学位。
\begin{education}
\item[2007年9月 --- 2010年6月] 南京大学计算机科学与技术系 \hfill 硕士
\item[2003年9月 --- 2007年6月] 南京大学计算机科学与技术系 \hfill 本科
\end{education}
% 论文作者在攻读学位期间所发表的文章的列表,按发表日期从近到远排列。
\begin{publications}
\item Xiaobao Wei, Jinnan Chen, ``Voting-on-Grid Clustering for Secure
  Localization in Wireless Sensor Networks,'' in \textsl{Proc. IEEE International
    Conference on Communications (ICC) 2010}, May. 2010.
\item Xiaobao Wei, Shiba Mao, Jinnan Chen, ``Protecting Source Location Privacy
  in Wireless Sensor Networks with Data Aggregation,'' in \textsl{Proc. 6th
    International Conference on Ubiquitous Intelligence and Computing (UIC)
    2009}, Oct. 2009.
\end{publications}
% 论文作者在攻读学位期间参与的科研课题的列表,按照日期从近到远排列。
\begin{projects}
\item 国家自然科学基金面上项目``问题研究''
(课题年限~2010年1月 --- 2012年12月),负责相关问题的研究。
\end{projects}
\end{resume}

%%%%%%%%%%%%%%%%%%%%%%%%%%%%%%%%%%%%%%%%%%%%%%%%%%%%%%%%%%%%%%%%%%%%%%%%%%%%%%%
% 生成版权及论文原创性说明
\statement

%%%%%%%%%%%%%%%%%%%%%%%%%%%%%%%%%%%%%%%%%%%%%%%%%%%%%%%%%%%%%%%%%%%%%%%%%%%%%%%
% 生成《学位论文出版授权书》页面,应放在最后一页
\makelicense

%%%%%%%%%%%%%%%%%%%%%%%%%%%%%%%%%%%%%%%%%%%%%%%%%%%%%%%%%%%%%%%%%%%%%%%%%%%%%%%
\end{document}
